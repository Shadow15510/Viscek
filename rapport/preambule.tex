% General packages
\usepackage{amsfonts, amssymb, amsthm}
\usepackage{array}
\usepackage{enumitem}
\usepackage{mathtools}
\usepackage{stmaryrd}
\usepackage{empheq, environ} % packages for system environment
\usepackage{calrsfs} % better mathcal letters
\usepackage{graphicx}
\usepackage{minted}


% Layout
\usepackage{fancyhdr, fancyvrb}
\usepackage{lastpage}
\usepackage[left=2.0cm, right=2.0cm, top = 2.0cm, bottom = 2.0cm]{geometry}
\usepackage{lettrine, yfonts}
\usepackage{multicol}
\usepackage{minitoc}
\usepackage{hyperref}
\hypersetup{hyperindex=true, colorlinks, linkcolor=blue, urlcolor=blue, citecolor=blue, breaklinks=true}
\everymath{\displaystyle}


% Language
\usepackage[utf8]{inputenc}
\usepackage[T1]{fontenc}
\usepackage{babel}


% Minted
\newcommand{\python}{\mintinline[breaklines=true, breakanywhere=true]{python}}
\newminted{python}{
	linenos=true,
	tabsize=4,
	breaklines=true,
	fontfamily=courier,
	autogobble,
	style=rainbow_dash,
	xleftmargin=5pt,
	xrightmargin=5pt,
	frame=lines
}


% Pagestyle
\pagestyle{fancy}
\fancypagestyle{plain}

\makeatletter
	\lhead{}
	\rhead{\@title}

	\lfoot{\@author}
	\cfoot{\thepage\ $|$ \pageref{LastPage}}
\makeatother



% Customs commands and environnements
\renewcommand{\leq}{\leqslant}
\renewcommand{\geq}{\geqslant}
\renewcommand{\ker}{\mathrm{Ker\,}}
\renewcommand{\vec}{\overrightarrow}
\newcommand{\im}{\mathrm{Im\,}}
\newcommand{\derivative}[2]{\frac{\mathrm{d} #1}{\mathrm{d}#2}}
\newcommand{\enluminure}[2]{\lettrine[lines=3]{\small \initfamily #1}{#2}}
\newcommand{\nnchapter}[1]{
	\chapter*{#1}
	\addstarredchapter{#1}
	\markboth{\uppercase{#1}}{}		
}

\def\restriction#1#2{\mathchoice
              {\setbox1\hbox{${\displaystyle #1}_{\scriptstyle #2}$}
              \restrictionaux{#1}{#2}}
              {\setbox1\hbox{${\textstyle #1}_{\scriptstyle #2}$}
              \restrictionaux{#1}{#2}}
              {\setbox1\hbox{${\scriptstyle #1}_{\scriptscriptstyle #2}$}
              \restrictionaux{#1}{#2}}
              {\setbox1\hbox{${\scriptscriptstyle #1}_{\scriptscriptstyle #2}$}
              \restrictionaux{#1}{#2}}}
\def\restrictionaux#1#2{{#1\,\smash{\vrule height .8\ht1 depth .85\dp1}}_{\,#2}}

\NewEnviron{system}[1][2]
{
	\begin{empheq}[left=\empheqlbrace]{alignat=#1}
        \BODY
    \end{empheq}
}
\NewEnviron{subsystem}[1][2]
{ 
    \begin{subequations}
    \begin{empheq}[left=\empheqlbrace]{alignat=#1}
    	\BODY
    \end{empheq}
    \end{subequations}
}
   
\newtheorem{theoreme}{Théorème}[section]
\newtheorem{definition}[theoreme]{Définition}
\newtheorem{proposition}[theoreme]{Proposition}
\newtheorem{propriete}[theoreme]{Propriété}
\newtheorem{lemme}[theoreme]{Lemme}
\newtheorem{formule}[theoreme]{Formule}
\newtheorem{remarque}[theoreme]{Remarque}
\newtheorem{exemple}[theoreme]{Exemple}
